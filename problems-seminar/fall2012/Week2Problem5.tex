\documentclass[fleqn]{article}
\usepackage{amsmath}
\usepackage{amssymb}
\usepackage[parfill]{parskip}
\usepackage[vmargin=0.5cm]{geometry}
\setlength\parindent{0pt}
\setlength{\mathindent}{0pt}
\begin{document}
1985 people attend a conference.

Fact 1. Given any 3 people at the conference, 
at least 2 speak a common language.

Fact 2. Everyone speaks at most 5 languages.

WTS that there is a language that at least 200 attendees speak.

Let $B_a = $ the set of attendees who do not share a common language with $a$.

Assume the contrary. 
Assumption 1. There is no grouping of 200 attendees such that they speak the same language.

Let's take one attendee, $a$.\\
By Fact 2, $a$ knows at most 5 languages.\\
By Assumption 1, for each language $a$ knows, there can be no more than 200 other attendees who also know it $\Rightarrow$ a shares a language with at most $199*5=995$ attendees. Thus, $|B_a| \geq 989$.

Let $a' \in B_a$.\\
Also by Fact 2 and Assumption 1, $|B_a'| \geq 989$.\\
By Assumption 1, we know 
$\exists a'' \in B_a' \setminus \lbrace a \rbrace, a'' \in B_a$.
(We can take a random group of 199 attendees from $B_a' \setminus \lbrace a \rbrace$, add $a$, and from Assumption 1 find an $a''$ who doesn't share a language with $a$)

However, by Fact 1, two attendees in the group 
$\lbrace a, a', a'' \rbrace$ must share a language 
$\Rightarrow \Leftarrow$

Thus, there is a language that at least 200 attendees speak.

\end{document}
