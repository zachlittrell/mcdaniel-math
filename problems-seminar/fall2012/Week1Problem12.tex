\documentclass[fleqn]{article}
\usepackage{amsmath}
\usepackage{amssymb}
\usepackage{tikz}
\usepackage[parfill]{parskip}
\usepackage[vmargin=0.5cm]{geometry}
\setlength\parindent{0pt}
\setlength{\mathindent}{0pt}
\begin{document}
There are five mathematicians at a conference: Hamblen, Ko, More, Naumov, and Simonelli.

Facts:
\begin{enumerate}
  \item  Each mathematician goes to sleep exactly twice.

  \item  For each pair of mathematicians, there is some time where
         they are both asleep.
\end{enumerate}

We want to show there is a point where three mathematicians are all sleeping at the same time.

First, we are going to construct an undirected graph $G$.

Rules for constructing $G$:
\begin{enumerate}
  \item For each mathematician, $M$, make two vertices, one for each
        time $M$ takes a snooze. We'll denote these snooze sessions
        as $M_1$ and $M_2$.
  \item Two vertices share an edge iff the snooze sessions the vertices
        represent overlap.
  \item Clearly, given a vertex $v$, $v$ does not share an edge with
        itself.
  \item Clearly, given a mathematician $M$, $M_1$ does not share an
        edge with $M_2$.
\end{enumerate}

To solve our problem, it suffices to show that there are 3 vertices in our graph $G$ who each share an edge with one another.

Definitions:
\begin{description}
  \item[$\sim$] For two vertices, $v$ and $w$, we define $v \sim w$
                to be true when $v$ shares an edge with $w$.
  \item[Simple Circuit] A sequence of distinct vertices make a 
                        simple circuit if you can traverse along
                        all the vertices in order and 
                        return to the first vertex.
  \item[Tree] An undirected graph with no simple circuits.
  \item[$V_G$] The vertices in graph $G$.
\end{description}

From how we construct our graph, we know there are 10 vertices.

Since we have 5 mathematicians, by Fact 2, for every pair of distinct mathematicians, $M$ and $N$, there must exist $i,j \in \lbrace 1,2 \rbrace$ such that $M_i \sim N_j$.

So, the number of edges in our graph must be at least ${5 \choose 2} = 10$.

Note the following theorem:

$T$ is a tree with $n$ vertices $\Rightarrow T$ has $n-1$ edges.

By the contrapositive, we can deduce that $G$ is not a tree, thus $G$ has at least one simple circuit.

Now, we'd like to prove the following proposition:\\
$P(n)$ is true $\Leftrightarrow$ if there is a simple circuit,$C$, of length $n$ in $V_G$, then $\exists$ distinct $ L,M,N \in C,
                        L \sim M \wedge M \sim N \wedge L \sim N$.\\
where $n \geq 3$.

Proof of $P$ by induction on $n$

Base Case: Let $n = 3$\\
Let $\lbrace L, M, N \rbrace$ be the sequence of vertices who make up a 3-length simple circuit in $V_G \Rightarrow\\
L \sim M \wedge M \sim N \wedge N \wedge L$.

Inductive Case: Let $n \geq 3$\\
Assume $P(n)$ is true.\\
Let $C$ be a sequence of vertices who make up an $n+1$-length simple circuit in $V_G$.\\
Let $\lbrace L,M \rbrace$ = the first two vertices in $C$.\\
Let $R$ = the vertices after $L$ and $M$ in $C$.\\
By the definition of $C$, $L \sim M \Rightarrow$\\
Let $N$ = a new vertex for the spans of time represented by $L$ and $M$ combined.\\
Construct a new graph, $G'$, with all vertices of $G$ except$L$ and $M$, and also with $N$, using rules 2-4 that we used for $G$.\\
Clearly, $\forall v \in V_G, v \in V_{G'} \wedge (v \sim L \vee v \sim M)
          \Rightarrow v \sim N$.\\
So, $N$ and $C$ make up an $n$-length simple circuit, which by $P(n)$ implies that $\exists$ distinct $X,Y,Z \in V,
              X \sim Y \wedge Y \sim Z \wedge Z \sim X$.\\
We need to make sure, however, that $X,Y,Z$ all correspond to snooze sessions in $G$, and thus to our five mathematicians.

Case 1. $N \not= X \wedge N \not= Y \wedge N \not= Z \Rightarrow
        X,Y,Z \in V_G$

Case 2. Without loss of generality, let $X = N$.\\
        $Y \sim N \Rightarrow Y \sim L \vee Y \sim M$\\
        Similarly, $Z \sim N \rightarrow Z \sim L \vee Z \sim M$.

Thus, $P(n+1)$ is true, and by mathematical induction, $P$ is true.

Thus, there is some point when three mathematicians are all sleeping at the same time.


\end{document}
