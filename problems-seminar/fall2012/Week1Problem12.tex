\documentclass[fleqn]{article}
\usepackage{amsmath}
\usepackage{amssymb}
\usepackage{tikz}
\usepackage[parfill]{parskip}
\usepackage[vmargin=0.5cm]{geometry}
\setlength\parindent{0pt}
\setlength{\mathindent}{0pt}

%macros

\newcommand{\triple}[3]{\langle #1,#2,#3 \rangle}
\newcommand{\set}[1]{\lbrace #1 \rbrace}
\newcommand{\buildset}[2]{\set{#1 | #2}}
\newcommand{\iunion}{\oplus}
\newcommand{\iinter}{\oslash}
\newcommand{\ioverlap}{\perp}

%tikz macros
\newcommand{\drawtext}[3]{
  \draw (#2,#3) node[minimum size=0.1cm,draw=none,fill=none,color=black]{#1};
}

\newcommand{\interval}[5]{
  \draw (#2,#3) node[]{} -- (#4,#3) node[]{};
  \drawtext{#1}{#5}{#3}
  }

\newcommand{\intervals}[1]{
  \begin{tikzpicture}
    [line width=0.1cm,blue,every node/.style={circle,minimum size=0.25cm,fill=blue}]
    #1
  \end{tikzpicture}
}

\begin{document}
There are five mathematicians at a conference: Hamblen, Ko, More, Naumov, and Simonelli.

\textbf{Facts}
\begin{enumerate}
  \item  Each mathematician goes to sleep exactly twice.

  \item  For each pair of mathematicians, there is some time where
         they are both asleep.
\end{enumerate}

We want to show there is a point where three mathematicians are all sleeping at the same time.

\textbf{Definitions}
\begin{itemize}
  \item \textbf{$M$} The set $\set{$Hamblen, Ko, More, Naumov, and Simonelli$}$
  \item \textbf{Snooze Interval} A Snooze Interval $s$ is a triplet of
        the form $\triple{name}{start}{end}$, where $start, end \in \mathbb{N}$ and $start \leq end$. $name$ is the identifier of who was snoozing.

For brevity, let $start_s = start$, $end_s = end$, and $name_s = name$.

        \intervals{
          \interval{$s$}{0}{0}{3}{5.5}
        }
  \item \textbf{$S$} The set of Snooze Intervals for each time each mathematician, $\mu$, in $M$, took a snooze.

  \item \textbf{$\ioverlap$} For two Snooze Intervals,
                         $s_1$ and
                         $s_2$, we define
                         $s_1 \ioverlap s_2$ as being true
                         when\\$name_{s_1} \not= name_{s_2}$ and $start_{s_1} \leq end_{s_2} \leq end_{s_1}$ or
                                $start_{s_1} \leq start_{s_2} \leq end_{s_1}$
                         or $s_2 \ioverlap s_1$.

  \item \textbf{$\iunion$} For two Snooze Intervals,
                         $s_1$ and
                         $s_2$, where $s_1 \ioverlap s_2$, we define
                         $s_1 \iunion s_2$ as \\
                         $\triple{name_{s_1}$ and $name_{s_2}}
                                 {min(start_{s_1}, start_{s_2})}
                                 {max(end_{s_1},end_{s_2})}$.

  \item \textbf{$\iinter$}
    For two Snooze Intervals,
    $s_1$ and
    $s_2$, we define\\
    $s_1 \iinter s_2 =
       \begin{cases}
         undefined &
           \text{if } s_1 \not\ioverlap s_2\\
         \triple{name_{s_1}$ over $name_{s_2}}{max(start_{s_1},end_{s_2})}
                                          {min(end_{s_2}, end_{s_1})} &
           \text{if } start_{s_1} \leq end_{s_2} \leq end_{s_1}\\
         \triple{name_{s_1}$ over $name_{s_2}}{max(start_{s_1},start_{s_2})}
                                   {min(start_{s_2},end_{s_1})} &
           \text{if } start_{s_1} \leq start_{s_2} \leq end_{s_1}
       \end{cases}$
  \item \textbf{Chain} A sequence of distinct Snooze Intervals $\set{s_1}$ is a chain if $s_1 \ioverlap s_2 \wedge s_2 \ioverlap s_3 \wedge ... s_{n-1} \ioverlap s_n$.
 
        \intervals{
          \interval{$s_1$}{0}{100}{2.25}{4}
          \interval{$s_2$}{2}{99.5}{3}{4}
          \interval{$s_3$}{2.55}{99}{3.25}{4}
          \interval{$s_4$}{1.5}{98.5}{3}{4}
          \drawtext{$\substack{.\\.\\.}$}{4}{98}
          \interval{$s_{n-1}$}{0.5}{97.5}{1.25}{4}
          \interval{$s_n$}{1}{97}{3}{4}
        }

  \item \textbf{Cycle} A chain $\set{s_1}$ is a cycle if $s_n \ioverlap s_1$.
       \end{itemize}

\textbf{Properties}
\begin{enumerate}
   \item Clearly, $\iunion$ is commutative.
   \item Clearly, $\iinter$ is commutative.
   \item Clearly, $\ioverlap$ is symmetric.
   \item $s_1 \ioverlap s_2 \wedge
          s_2 \ioverlap s_3 \wedge
          s_3 \ioverlap s_1 \Rightarrow
          s_1 \ioverlap (s_2 \iinter s_3)$

         \intervals{
           \interval{\textbf{$s_1$}}{0.5}{3}{2.75}{4}
           \interval{\textbf{$s_2$}}{1.75}{2.5}{3}{4}
           \interval{\textbf{$s_3$}}{1}{2}{2.5}{4}
         }

         \underline{Proof}
         Suppose the contrary.\\
         $s_1 \not\ioverlap (s_2 \iinter s_3).$
  \item $s_1 \ioverlap (s_2 \iunion s_3) \Rightarrow
         s_1 \ioverlap s_2 \vee s_1 \ioverlap s_3$
       
\end{enumerate}


\end{document}
