\documentclass[fleqn]{article}
\usepackage{amsmath}
\usepackage{amssymb}
\usepackage{tikz}
\usepackage[parfill]{parskip}
\usepackage[vmargin=0.5cm]{geometry}
\setlength\parindent{0pt}
\setlength{\mathindent}{0pt}

%macros

\newcommand{\triple}[3]{\langle #1,#2,#3 \rangle}
\newcommand{\set}[1]{\lbrace #1 \rbrace}
\newcommand{\buildset}[2]{\set{#1 | #2}}
\newcommand{\iunion}{\oplus}
\newcommand{\iinter}{\oslash}
\newcommand{\ioverlap}{\perp}

%tikz macros
\newcommand{\interval}[4]{
  \draw (#2,#3) node[]{} --(#2+#4,#3) node[]{};
  }

\newcommand{\intervalpic}[1]{
  \begin{tikzpicture}
    [scale=.8,auto=left,line width=0.1cm,blue!20,every node/.style={circle,minimum size=0.25cm,fill=blue!20}]
    #1
  \end{tikzpicture}
}

\begin{document}
There are five mathematicians at a conference: Hamblen, Ko, More, Naumov, and Simonelli.

\textbf{Facts}
\begin{enumerate}
  \item  Each mathematician goes to sleep exactly twice.

  \item  For each pair of mathematicians, there is some time where
         they are both asleep.
\end{enumerate}

We want to show there is a point where three mathematicians are all sleeping at the same time.

\textbf{Definitions}
\begin{itemize}
  \item \textbf{$M$} The set $\set{$Hamblen, Ko, More, Naumov, and Simonelli$}$
  \item \textbf{Snooze Interval} A Snooze Interval is a triplet of
        the form $\triple{snoozer}{start}{end}$, where $start, end \in \mathbb{N}$ and $start \leq end$. $snoozer$ is the identifier of who was snoozing.
  \item \textbf{$S$} The set of Snooze Intervals for each time each mathematician, $\mu$, in $M$, took a snooze.
  \item \textbf{$\iunion$} For two Snooze Intervals,
                         $\triple{snoozer_1}{start_1}{end_1}$ and
                         $\triple{snoozer_2}{start_2}{end_2}$, we define
                         $\triple{snoozer_1}{start_1}{end_1} \iunion
                          \triple{snoozer_2}{start_2}{end_2}$ as \\
                         $\triple{snoozer_1 and snoozer_2}
                                 {min(start_1, start_2)}
                                 {max(end_1,end_2)}$.

  \item \textbf{$\ioverlap$} For two Snooze Intervals,
                         $\triple{snoozer_1}{start_1}{end_1}$ and
                         $\triple{snoozer_2}{start_2}{end_2}$, we define
                         $\triple{snoozer_1}{start_1}{end_1} \ioverlap
                          \triple{snoozer_2}{start_2}{end_2}$ as being true
                         when\\$snoozer_1 \not= snoozer_2$ and $start_1 \leq end_2 \leq end_1$ or
                                $start_1 \leq start_2 \leq end_1$.

  \item \textbf{$\iinter$}
    For two Snooze Intervals,
    $s_1=\triple{snoozer_1}{start_1}{end_1}$ and
    $s_2=\triple{snoozer_2}{start_2}{end_2}$, we define
    $s_1 \iinter s_2 =
       \begin{cases}
         \triple{snoozer_1 over snoozer_2}{\infty}{\infty} &
           \text{if } s_1 \not\ioverlap s_2\\
         \triple{snoozer_1 over snoozer_2}{max(start_1,end_2)}
                                          {min(end_2, end_1)} &
           \text{if } start_1 \leq end_2 \leq end_1\\
         \triple{snoozer_1 over snoozer_2}{max(start_1,start_2)}
                                   {min(start_2,end_1)} &
           \text{if } start_1 \leq start_2 \leq end_1
       \end{cases}$
  \item \textbf{Chain} A sequence of distinct Snooze Intervals $\lbrace s_i \rbrace$ is a chain if $s_1 \ioverlap s_2 \wedge s_2 \ioverlap s_3 \wedge ... s_n \ioverlap s_1$.
\end{itemize}

\textbf{Properties}
\begin{enumerate}
  \item $s_1 \ioverlap s_2 \wedge
         s_2 \ioverlap s_3 \wedge
         s_3 \ioverlap s_1 \Rightarrow
         s_1 \ioverlap (s_2 \iinter s_3)$

        \underline{Proof}
        Suppose the contrary.\\
        $s_1 \not\ioverlap (s_2 \iinter s_3).$
   \item Clearly, $\iunion$ is commutative.
   \item Clearly, $\iinter$ is commutative.
   \item Clearly, $\ioverlap$ is symmetric.
\end{enumerate}


\end{document}
