\documentclass[fleqn]{article}
\usepackage{amsmath}
\usepackage{amssymb}
\usepackage{tikz}
\usepackage[parfill]{parskip}
\usepackage[vmargin=0.5cm]{geometry}
\setlength\parindent{0pt}
\setlength{\mathindent}{0pt}

%macros

%constructs a 3-tuple around its 3 arguments
\newcommand{\triple}[3]{\langle #1,#2,#3 \rangle}
%constructs a set around its argument
\newcommand{\set}[1]{\lbrace #1 \rbrace}
%constructs a set in the set-builder notation with the first
%argument before the bar and the second argument after.
\newcommand{\buildset}[2]{\set{#1 | #2}}
%constructs a closed interval between the two arguments
\newcommand{\interval}[2]{\lbrack #1, #2 \rbrack}
%union between intervals
\newcommand{\iunion}{\oplus}
%intersection between intervals
\newcommand{\iinter}{\oslash}
%overlap between intervals
\newcommand{\ioverlap}{\perp}

%tikz macros
%draws just text
\newcommand{\drawtext}[3]{
  \draw (#2,#3) node[minimum size=0.1cm,draw=none,fill=none,color=black]{#1};
}
%draws an interval line, where #1 is the label,
%#2 is the x position, #3 is the y position,
%#4 is the width, and #5 is the x position of the label.
\newcommand{\intervalline}[5]{
  \draw (#2,#3) node[]{} -- (#4,#3) node[]{};
  \drawtext{#1}{#5}{#3}
  }
%a wrapper around begin{tikzpicture} end{tikzpicture}, for how we'll
%draw intervals.
\newcommand{\intervals}[1]{
  \begin{tikzpicture}
    [line width=0.1cm,blue,every node/.style={circle,minimum size=0.25cm,fill=blue}]
    #1
  \end{tikzpicture}
}

\begin{document}
There are five mathematicians at a conference: Hamblen, Ko, More, Naumov, and Simonelli.

We want to show there is a point where three mathematicians are all sleeping at the same time.

\textbf{Facts}
\begin{enumerate}
  \item  Each mathematician goes to sleep exactly twice.

  \item  For each pair of mathematicians, there is some time where
         they are both asleep.
\end{enumerate}

\textbf{Definitions}
\begin{itemize}
  \item \textbf{$M$} The set $\set{$Hamblen, Ko, More, Naumov, and Simonelli$}$
  \item \textbf{Snooze Interval} A Snooze Interval $s$ is a triplet of
        the form $\triple{name}{start}{end}$, where $start, end \in \mathbb{N}$ and $start \leq end$. $name$ is the identifier of who was snoozing.

For brevity, let $start_s = start$, $end_s = end$, $name_s = name$, and $interval_s = \interval{start}{end}$.

        \intervals{
          \intervalline{$s$}{0}{0}{3}{5.5}
        }
  \item \textbf{$S$} The set of Snooze Intervals for each time each mathematician, $\mu$, in $M$, took a snooze.

  \item \textbf{$\ioverlap$} For two Snooze Intervals,
                         $s_1$ and
                         $s_2$, we define
                         $s_1 \ioverlap s_2$ as being true
                         when\\$name_{s_1} \not= name_{s_2}$ and $start_{s_1} \leq end_{s_2} \leq end_{s_1}$ or
                                $start_{s_1} \leq start_{s_2} \leq end_{s_1}$
                         or $s_2 \ioverlap s_1$.

  \item \textbf{$\iunion$} For two Snooze Intervals,
                         $s_1$ and
                         $s_2$, where $s_1 \ioverlap s_2$, we define
                         $s_1 \iunion s_2$ as \\
                         $\triple{name_{s_1}$ and $name_{s_2}}
                                 {min(start_{s_1}, start_{s_2})}
                                 {max(end_{s_1},end_{s_2})}$.

                         \intervals{
                           \intervalline{$s_1$}{0}{3}{2}{4}
                           \intervalline{$s_2$}{1}{2.5}{2.5}{4}
                           \intervalline{$s_1 \iunion s_2$}{0}{2}{2.5}{4}
                         }

  \item \textbf{$\iinter$}
    For two Snooze Intervals,
    $s_1$ and
    $s_2$, we define\\
    $s_1 \iinter s_2 =
       \begin{cases}
         undefined &
           \text{if } s_1 \not\ioverlap s_2\\
         \triple{name_{s_1}$ over $name_{s_2}}
                {max(start_{s_1},start_{s_2})}
                {end_{s_2}} &
           \text{if } start_{s_1} \leq end_{s_2} \leq end_{s_1}\\
         \triple{name_{s_1}$ over $name_{s_2}}
                {start_{s_2}}
                {min(end_{s_2},end_{s_1})} &
           \text{if } start_{s_1} \leq start_{s_2} \leq end_{s_1}
       \end{cases}$

        \intervals{
           \intervalline{$s_1$}{0}{3}{2}{4}
           \intervalline{$s_2$}{1}{2.5}{2.5}{4}
           \intervalline{$s_1 \iinter s_2$}{1}{2}{2}{4}
        }


  \item \textbf{Chain} A sequence of distinct Snooze Intervals $\set{s_1}$ is a chain if $s_1 \ioverlap s_2 \wedge s_2 \ioverlap s_3 \wedge ... s_{n-1} \ioverlap s_n$.
 
        \intervals{
          \intervalline{$s_1$}{0}{100}{2.25}{4}
          \intervalline{$s_2$}{2}{99.5}{3}{4}
          \intervalline{$s_3$}{2.55}{99}{3.25}{4}
          \intervalline{$s_4$}{1.5}{98.5}{3}{4}
          \drawtext{$\substack{.\\.\\.}$}{4}{98}
          \intervalline{$s_{n-1}$}{0.5}{97.5}{1.25}{4}
          \intervalline{$s_n$}{1}{97}{3}{4}
        }

  \item \textbf{Cycle} A chain $\set{s_1}$ is a cycle if $s_n \ioverlap s_1$.
       \end{itemize}

\textbf{Properties}
\begin{enumerate}
   \item Clearly, $\iunion$ is commutative.
   \item Clearly, $\iinter$ is commutative.
   \item Clearly, $\ioverlap$ is symmetric.
   \item $s_1 \ioverlap s_2 \wedge
          s_2 \ioverlap s_3 \wedge
          s_3 \ioverlap s_1 \Rightarrow
          s_1 \ioverlap (s_2 \iinter s_3)$

         \underline{Proof}\\
           Since $s_2 \ioverlap s_3$, $s_2 \iinter s_3$ is defined.\\
           Without loss of generality, suppose 
           $start_{s_2} \leq start_{s_3}$
           \begin{itemize}
             \item Case 1: $end_{s_2} \geq end_{s_3}$.

                   \intervals{
                     \intervalline{\textbf{$s_2$}}{0}{1}{2.5}{4}
                     \intervalline{\textbf{$s_3$}}{1}{0.5}{2}{4}
                   }

                   Then $interval_{s_2 \iinter s_3} = interval_{s_3} 
                        \Rightarrow s_1 \ioverlap s_2 \iinter s_3$
             \item Case 2: $end_{s_2} \leq end_{s_3}$.

                   \intervals{
                     \intervalline{\textbf{$s_2$}}{0}{1}{2.5}{4}
                     \intervalline{\textbf{$s_3$}}{1}{0.5}{3}{4}
                   }

                   Suppose $s_1 \not\ioverlap s_2 \iinter s_3$\\
                   Then $start_{s_1} > end_{s_2}$ and
                        $end_{s_1} < start_{s_3} \Rightarrow
                        end_{s_1} < start_{s_1} \Rightarrow\Leftarrow$\\
                   Thus, $s_1 \ioverlap s_2 \iinter s_3$.
           \end{itemize}
                  

  \item $s_1 \ioverlap (s_2 \iunion s_3) \Leftrightarrow
         s_2 \ioverlap s_3 \wedge
         (s_1 \ioverlap s_2 \vee s_1 \ioverlap s_3)$
        \begin{itemize}
          \item \underline{Proof for $\Leftarrow$}
                 Without loss of generality, assume $s_1 \ioverlap s_2$.\\
                 Clearly, $s_1 \ioverlap s_2 \iunion s_3$.
                   
          \item \underline{Proof for $\Rightarrow$}
                 Without loss of generality, assume $start_{s_2} \leq start_{s_3}$.\\
                 Clearly, $s_2 \ioverlap s_3$.
                 \begin{itemize}
                   \item Case 1: $end_{s_2} \geq end_{s_3}$

                            \intervals{
                              \intervalline{\textbf{$s_2$}}{0}{1}{2.5}{4}
                              \intervalline{\textbf{$s_3$}}{1}{0.5}{2}{4}
                            }

                         Then $interval_{s_2 \iunion s_3} = interval_{s_2} \Rightarrow
                         s_1 \ioverlap s_2$
                   \item Case 2: $end_{s_2} \leq end_{s_3}$

                            \intervals{
                              \intervalline{\textbf{$s_2$}}{0}{1}{2.5}{4}
                              \intervalline{\textbf{$s_3$}}{1}{0.5}{3}{4}
                            }

                         \begin{itemize}
                           \item Case I: $start_{s_1} \leq end_{s_2} \Rightarrow
                                 s_1 \ioverlap s_2$
                           \item Case II: $start_{s_1} \geq end_{s_2} \Rightarrow
                                 s_1 \ioverlap s_3$. 
                         \end{itemize}                
          \end{itemize}
        \end{itemize}
\end{enumerate}
By Fact 2, we know there must be at least ${5 \choose 2}=10$ distinct overlapping pairs of mathematicians.\\
We also have 10 Snooze Intervals in $S$, so we'd like to prove that given these numbers, there must be a Cycle of Snooze Intervals.

\begin{itemize}
  \item \textbf{Lemma 1}\\
        If set of Snooze Intervals $S$ has $n$ elements and no cycles,
        then $S$ has at most $n-1$ distinct overlapping pairs.

        \underline{Proof of Lemma 1 by Induction on $n$}
        \begin{itemize}
          \item Base Case: Let $n$ = 1.\\
                Then there are 0 overlapping pairs, thus
                Lemma 1 is true for $n=1$.
          \item Inductive Case: Let $n \geq 1$\\
                Assume Lemma 1 is true for all x such that $1 \leq x \leq n$

                Let $S$ be a set of $n+1$ Snooze Intervals.\\
                Choose a Snooze Interval $s$ in $S$.

                \begin{itemize}
                  \item Case 1. $\forall \sigma \in S, s \not\ioverlap \sigma$\\
                        Clearly, $S \setminus \set{s}$ has no cycles.\\
                        By Lemma 1, $S \setminus \set{s}$ has at most 
                        $n-1$ distinct overlapping pairs $\Rightarrow\\
                        S$ has at most $n-1$ distinct overlapping pairs.
                  \item Case 2. $\exists \sigma \in S, s \ioverlap \sigma$\\
                        Since $S$ has no cycles and $S$ is finite, we
                        can construct a Chain starting at $s$, until
                        we find a Snooze Interval $\lambda$ who only
                        overlaps with one other Snooze Interval.\\
                        (If this was not true, we'd either find an
                         infinite number of Snooze Intervals, or 
                         overlap with a Snooze Interval already in 
                         the Chain, thus creating a Cycle).\\
                        Clearly, $S \setminus \set{\lambda}$ has no cycles.\\
                        By Lemma 1, $S \setminus \set{\lambda}$ has at
                        most $n-1$ distinct overlapping pairs $\Rightarrow\\
                        S$ has at most $n$ distinct overlapping pairs.
                        
                        Thus, Lemma 1 holds for all $n$.
                \end{itemize}
        \end{itemize}

\end{itemize}
By the contrapositive of Lemma 1, we can deduce that there is at least one Cycle in $S$. 

To solve our original problem, it suffices to show that the following proposition:

$P(n)="$ if there is a Cycle $C$ of length $n \geq 3$, then $\exists$ distinct $s_1, s_2, s_3 \in C, s_1 \ioverlap (s_2 \iinter s_3)$."

\underline{Proof of $P$ by induction on $n$}
\begin{itemize}
  \item Base Case: Let $n = 3$\\
        Let $\set{s_1,s_2,s_3}$ be a 3-length Cycle $\Rightarrow\\
        s_1 \ioverlap s_2 \wedge s_2 \ioverlap s_3 \wedge s_3 \ioverlap s_1$.\\
        By Property 4, $s_1 \ioverlap (s_2 \iinter s_3)$.
  \item Inductive Case: Let $n \geq 3$\\
        Assume $P(n)$ is true.\\
        Let $\set{s_1,s_2...s_n,s_{n+1}}$ be an $n+1-$length cycle.\\
        By definition of Cycles, $s_1 \ioverlap s_2$.\\
        Let $C' = \set{s_1 \iunion s_2, s_3,...s_n, s_{n+1}}$\\
        By Property 5, $s_3 \ioverlap s_1 \iunion s_2$ and 
                       $s_{n+1} \ioverlap s_1 \iunion s_2$.\\
        So, $C'$ is a Cycle as well.\\
        By $P(n)$, $\exists$ distinct $x,y,z \in C', x \ioverlap (y \iinter z)$.

        \begin{itemize}
          \item Case 1. $s_1 \iunion s_2 \not= x,y $ or $z$. \\
                Then $x,y,z \in C$
          \item Case 2. Without less of generality, let 
                $x = s_1 \iunion s_2$.\\
                Then $(y \iinter z) \ioverlap (s_1 \iunion s_2)$.\\
                By Property 5, $s_1 \ioverlap (y \iinter z)$ or
                               $s_2 \ioverlap (y \iinter z)$.\\
                Thus, either $\set{s_1,y,z}$ or $\set{s_2,y,z}$ form
                the triplet we seek.
        \end{itemize}
        
        Thus, $P$ is true for all $n$.
        
\end{itemize} 
       
Thus, there must be 3 mathematicians who are all sleeping at the same time at some point.


\end{document}
