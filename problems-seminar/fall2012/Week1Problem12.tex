\documentclass[fleqn]{article}
\usepackage{amsmath}
\usepackage{amssymb}
\usepackage[parfill]{parskip}
\usepackage[vmargin=0.5cm]{geometry}
\setlength\parindent{0pt}
\setlength{\mathindent}{0pt}
\begin{document}
There are five mathematicians at a conference: Hamblen, Ko, More, Naumov, and Simonelli.

\textbf{Facts}
\begin{enumerate}
  \item  Each mathematician goes to sleep exactly twice.

  \item  For each pair of mathematicians, there is some time where
         they are both asleep.
\end{enumerate}

We want to show there is a point where three mathematicians are all sleeping at the same time.

First, we are going to construct an undirected graph $G$.

\textbf{Rules for constructing $G$}:
\begin{enumerate}
  \item For each mathematician, $M$, make two vertices, one for each
        time $M$ takes a snooze. We'll denote these snooze sessions
        as $M_1$ and $M_2$.
  \item Two vertices share an edge iff the snooze sessions the vertices
        represent overlap.
  \item Clearly, given a vertex $v$, $v$ does not share an edge with
        itself.
  \item Clearly, given a mathematician $M$, $M_1$ does not share an
        edge with $M_2$.
\end{enumerate}

\textbf{Definitions}
\begin{itemize}
  \item \textbf{Simple Circuit}  A sequence of distinct vertices make a 
                        simple circuit if you can traverse along
                        all the vertices in order and 
                        return to the first vertex.
  \item \textbf{$V_G$} The vertices in graph $G$.
  \item \textbf{Snooze Vertex} A vertex that represents a span of time.
  \item \textbf{$time(v)$} The span of time Snooze Vertex $v$ 
                           represents.
  \item \textbf{$\oslash$} For two Snooze Vertices, $v$ and $w$,
                           we define $v \oslash w$ as a Snooze Vertex
                           that represents the span of time where
                           $time(v)$ and $time(w)$ overlaps.
  \item \textbf{$\oplus$}  For two Snooze Vertices, $v$ and $w$,
                           we define $v \oplus w$ as a Snooze Vertex
                           that represents $time(v)$ and $time(w)$
                           combined. 
  \item \textbf{$\circledast$} For two Snooze Vertices, $v$ and $w$,
                               we defined $v \circledast w$ to be true
                               if $time(v)$ and $time(w)$ overlap.
                               If $time(v)$ or $time(w)$ are 0-length
                               spans of time, then $v \circledast w$
                               is false.
\end{itemize}

\textbf{Properties}
\begin{enumerate}
  \item Given three snooze vertices, $v$,$w$, and $x$,\\
        $v \circledast w \wedge 
         v \circledast x \wedge 
         x \circledast w \Leftrightarrow 
         v \circledast (x \oslash w)$.

        Proof:\\
        $\Rightarrow$\\
        Suppose the contrary is true.\\
        Then there is $time(v)$ overlaps with $time(w)$,
        and there's a $time(x)$ overlaps with $time(w)$ as well.\\
        We also know that $time(v)$ overlaps with $time(x)$.\\
        But, since $\neg(v \circledast (x \oslash w))$, then either
        $v$ doesn't overlap with both or either, or 
        $v$'s snooze session would have to pause during
        $time(x \oslash w) \Rightarrow \Leftarrow$.

         $\Leftarrow$\\
         If $time(v)$ overlaps where $time(x)$ overlaps with $time(w)$,
         then clearly each span of time overlaps with each  other.
  \item Clearly, $\oslash$ is commutative.
  \item Clearly, $\oplus$ is commutative.
  \item Clearly, $\circledast$ is symmetrical.
\end{enumerate}

From how we construct our graph, we know there are 10 vertices.

Since we have 5 mathematicians, by Fact 2, for every pair of distinct mathematicians, $M$ and $N$, there must exist $i,j \in \lbrace 1,2 \rbrace$ such that $M_i \circledast N_j$.

So, the number of edges in our graph must be at least ${5 \choose 2} = 10$.

First, we're going to define and prove the following theorem:

\begin{itemize}
\item
\textbf{Theorem 1}\\
If graph $G$ has $n$ vertices and no simple circuits,
then $G$ has at most $n-1$ edges.

Proof of Theorem 1 by Induction on $n$:

Base Case: Let $n = 1$.\\
Then there are 0 edges, thus Theorem 1 works for $n=1$.

Inductive Case: Let $n \geq 1$\\
Assume Theorem 1 is true for all $x$ such that $1 \leq x \leq n$

Let $G$ be a graph with $n+1$ vertices.\\
We want to show that $G$ has at most $n$ edges.

Choose a vertex $v$ in $G$.

Case 1. $v$ has no edges.\\
We then create a graph, $G'$, with $v$ removed.\\
Obviously, $G'$ also has no simple circuits, and by our inductive hypothesis, $G'$ has at most $n-1$ edges, which means $G$ also has at most $n-1$ edges.

Case 2. $v$ has at least one edge.\\
Since $G$ has no circuits, we can follow a path of edges from $v$ until we arrive at a vertex, $l$ with no other edge besides the one used to get to it (if this was not the case, we could either find an infinite number of distinct vertices, but $G$ is finite, or end up at a vertex we visited before, thus creating a simple circuit).\\
We then create a graph called $G'$ with $l$ and its edge removed.
\\Clearly, $G'$ has no simple circuits, and by our inductive hypothesis, $G'$ has at most $n-1$ edges. So $G$ has at most $n-1+1=n$ edges.


So Theorem 1 is true for $n+1$, and by mathematical induction, Theorem 1 is true for all $n$.
\end{itemize} 
By the contrapositive of Theorem 1, we can deduce that $G$ has at least one simple circuit.

To solve our problem, we'd like to find that regardless of how long the simple circuit in $G$ is, we can find three Snooze Vertices that overlap with one another.\\
So, it suffices to prove the following proposition:\\
For a graph, $G$\\
$P(n)$ is true $\Leftrightarrow$ if there is a simple circuit,$C$, of length $n$ in $V_G$,\\ then $\exists$ distinct $ L,M,N \in C,
                       L \circledast (M \oslash N)$ .\\
where $n \geq 3$.

Proof of $P$ by induction on $n$

Base Case: Let $n = 3$\\
Let $\lbrace L, M, N \rbrace$ be the sequence of vertices who make up a 3-length simple circuit in $V_G \Rightarrow\\
L \circledast M \wedge M \circledast N \wedge N \circledast L$.\\
By Property 1, $L \circledast (M \oslash N)$.

Inductive Case: Let $n \geq 3$\\
Assume $P(n)$ is true.\\
Let $C$ be a sequence of vertices who make up an $n+1$-length simple circuit in $V_G$.\\
Let $\lbrace L,M \rbrace$ = the first two vertices in $C$.\\
Let $R$ = the vertices after $L$ and $M$ in $C$.\\
By the definition of $C$, $L \circledast M$\\
Let $N$ = $L \oplus M$.
Construct a new graph, $G'$, with all vertices of $G$ except$L$ and $M$, and also with $N$, using rules 2-4 that we used for $G$.\\
Clearly, $\forall v \in V_G, v \in V_{G'} \wedge 
                             (v \circledast L \vee v \circledast M)
          \Rightarrow v \circledast N$.\\
So, $N$ and $C$ make up an $n$-length simple circuit, which by $P(n)$ implies that\\
$\exists$ distinct $X,Y,Z \in V_{G'},
              Y \circledast Z \wedge
              X \circledast (Y \oslash Z)$

Now, using $X,Y,Z$, we need to find three vertices in $V_G$ that are similar interconnecting.

Cases
\begin{enumerate}
  \item $N \not= X \wedge N \not= Y \wedge N \not= Z \Rightarrow
        X,Y,Z \in V_G$,\\
        so $X,Y,Z$ form the triplet we're looking for.

  \item Without loss of generality, let $X = N$.\\
        Which means $(L \oplus M) \circledast (Y \oslash Z) \Rightarrow$\\
        So, there is some span of time where $Y \oslash Z$ overlaps
        with either $L$ or $M$, or both.\\
        From this, we can get our triplet of vertices.
\end{enumerate}
Thus, $P(n+1)$ is true, and by mathematical induction, $P$ is true.

Thus, there is some point when three mathematicians are all sleeping at the same time.


\end{document}
