\documentclass{article}
\usepackage{amsmath}
\usepackage{amssymb}
\begin{document}

\begin{center}\item \section*{Problem}\end{center}

Given a permutation $\sigma$ in $S_n$, $\sigma(i)$ is a $large$ integer if $\sigma(i) > \sigma(j)$, for all $i < j \leq n$. Find the average number of large integers over all permutations in $S_n$.

\begin{center}\item \section*{Solution}\end{center}

Let $n \geq 1$. To compute the average number of large integers over all permutations in $S_n$, we will find the sum of the number of times index $i$ maps to a large integer divided by $n!$, for each $i$.

Suppose $\sigma \in S_n$ is a permutation where $\sigma(i)$ is a large integer.

Choose $i-1$ integers for indices $1,2..i-1$, and call the set of these choices $L$. By the largeness of $\sigma(i)$, $\sigma(i)$ must be 
$max(\{1..n\}\setminus L)$.
Let $U = \{1,2..n\}\setminus L \setminus \{\sigma(i)\}$.
As we will map indices $i+1...n$ to elements of $U$, we have $|U|!=(n-i)!$
choices for those indices.

To get the average of the possible combinations, we first choose the first $i-1$ indices, ${n\choose {i-1}}$, multiply by the number of ways we can arrange them, $(i-1)!$, and multiply that by $(n-i)!$ choices for the last $n-i-1$ indices, and divide this by $n!$.

\begin{align*}
    \frac{{n\choose {i-1}}(i-1)!(n-i)!}
         {n!}
  &=\frac{\frac{n!}
               {(n-i+1)!(i-1)!}(i-1)!(n-i)!}
         {n!}
\\&=\frac{\frac{n!(n-i)!}
               {(n-i+1)!}}
         {n!}
\\&=\frac{(n-i)!}
         {(n-i+1)!}
\\&=\frac{1}
         {n-i+1}
\end{align*}

Thus, when we add the average for each index, we get that the average number of large integers over all permutations in $S_n$ is 
$\displaystyle\sum\limits_{i=1}^n \frac{1}{i}$. 

\end{document}
