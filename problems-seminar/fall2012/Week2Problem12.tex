\documentclass{article}
\usepackage{float}
\usepackage{tikz}
\usetikzlibrary{shapes}
\usetikzlibrary{plotmarks}
\newcommand{\drawchessboard}[1]{
  \draw [very thin,step=#1] (0,0) grid(#1*8,#1*8);
}
\newcommand{\drawmidpoint}[3]{
  \draw [very thin] plot [mark=*] coordinates{(#1*#3-0.5*#3,#2*#3-0.5*#3)};
}
\newcommand{\drawmidpoints}[1]{
    \foreach \i in {1,...,8}{
      \drawmidpoint{1}{\i}{#1}
      \drawmidpoint{8}{\i}{#1}
      \drawmidpoint{\i}{1}{#1}
      \drawmidpoint{\i}{8}{#1}
    }
}
\newcommand{\drawmidpointlines}[1]{
  \draw (0.5*#1,0.5*#1)--(7.5*#1,0.5*#1);
  \draw (0.5*#1,7.5*#1)--(7.5*#1,7.5*#1);
  \draw (0.5*#1,0.5*#1)--(0.5*#1,7.5*#1);
  \draw (7.5*#1,0.5*#1)--(7.5*#1,7.5*#1);
}

\newcommand{\highlightsegment}[5]{
  \draw[thick, red] (#2*#1,#3*#1)--(#4*#1,#5*#1);
}

\newcommand{\horizontalcut}[4]{
%  \highlightsegment{#1}{0.5}{#2}{0.5}{#3}
%  \highlightsegment{#1}{7.5}{#2}{7.5}{#3}
  \highlightsegment{#1}{-0.25}{#4}{8.25}{#4}
}

\newcommand{\verticalcut}[4]{
%  \highlightsegment{#1}{#2}{0.5}{#3}{0.5}
%  \highlightsegment{#1}{#2}{7.5}{#3}{7.5}
  \highlightsegment{#1}{#4}{-0.25}{#4}{8.25}
}

\newcommand{\badpiece}[4]{
  \filldraw[opacity=0.5,fill=blue!40](#1,#2) rectangle ++(#3,#4);
}

\newcommand{\chessboard}[2]{
  \begin{figure}[H]
    \centering
    \begin{tikzpicture}
      \drawchessboard{#1}
      #2
    \end{tikzpicture}
  \end{figure}
}


\begin{document}
\begin{center}\item \section*{Problem}\end{center}

Can we cut an $8 \times 8$ chessboard with 13 straight lines, none passing through the midpoint of a square, such that every resulting piece contains at most one midpoint of a square?

\begin{center}\item \section*{Solution}\end{center}

First, note that our chessboard will have 28 squares along its perimeter.

\chessboard{0.5}{
  \drawmidpoints{0.5}
}

Consider each of these 28 midpoints being connected to their adjacent neighbors by straight line segments.

\chessboard{0.5}{
  \drawmidpoints{0.5}
  \drawmidpointlines{0.5}
}

By our assumption, any cut across the board cannot go through the midpoints of these squares. 

Also, note that since the cuts are straight, they can only intersect with either two line segments or none (the cut may be along the edge of the board past the midpoints).


\chessboard{0.5}{
  \drawmidpointlines{0.5}
  \verticalcut{0.5}{1.5}{2.5}{2.2}
  \drawmidpoints{0.5}
}


Thus, for our 13 cuts, at most 26 segments will be intersected. Thus there exists a segment that isn't intersected. The two midpoints this line segment connects thus both are part of the same contiguous piece, which contains more than one midpoint.

\chessboard{0.5}{
  \drawmidpointlines{0.5}
  \verticalcut{0.5}{1.5}{2.5}{2.2}
  \verticalcut{0.5}{2.5}{3.5}{3.2}
  \verticalcut{0.5}{3.5}{4.5}{4.1}
  \verticalcut{0.5}{4.5}{5.5}{4.8}
  \verticalcut{0.5}{5.5}{6.5}{5.9}
  \verticalcut{0.5}{6.5}{7.5}{7.0}
  \horizontalcut{0.5}{6.5}{7.5}{7.0}
  \horizontalcut{0.5}{6.5}{5.5}{6.2}
  \horizontalcut{0.5}{5.5}{4.5}{4.7}
  \horizontalcut{0.5}{4.5}{3.5}{3.9}
  \horizontalcut{0.5}{3.5}{2.5}{3.0}
  \horizontalcut{0.5}{2.5}{1.5}{2.1}
  \horizontalcut{0.5}{1.5}{0.5}{0.8}
  \drawmidpoints{0.5}
  \badpiece{0}{3.5}{1.1}{0.5}
  \badpiece{0}{0}{1.1}{0.4}
}

Thus, we \textbf{cannot} cut our chessboard as the problem requires.


\end{document}
